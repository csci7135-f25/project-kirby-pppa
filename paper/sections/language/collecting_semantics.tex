\section{Adversarial Equivalence and Collecting Semantics}

In this section we ultimately define when two \lang protocols are adversarially equivalent with respect to a given adversary $\mathcal{A}$.
This definition relies on a collecting semantics for \lang protocols and adversaries (Section~\ref{subsec:collecting-semantics}) and the notion of adversarial interaction properties (Section~\ref{subsec:adversarial-interaction-properties}).

\subsection{Adversarial Interaction Properties}
\label{subsec:adversarial-interaction-properties}

Adversarial equivalence is a relational property defined on two sets of adversarial interaction properties.
Adversarial interaction properties are designed to capture sufficient information for two purposes: \textcircled{1} to model the interactions and observations an adversary has with a protocol and any information it can derive from those interactions, and \textcircled{2} to model system execution.
The first purpose is necessary to reason about what an adversary has learned from interacting with a protocol, while the second purpose is necessary to reason about possible system executions that produce the relevant interactions, observations, and knowledge.

\begin{definition}[Adversarial Interaction Property]
  \label{def:adversarial-interaction-property}
  An adversarial interaction property is a tuple of the form $(\alpha, \kappa, N, \Sigma)$ where $\alpha$ is a sequence of adversary interactions with and observable effects from a protocol, $\kappa$ is a sequence of values representing the adversary's current knowledge, $N$ is the current network state, and $\Sigma$ is the current protocol state.
\end{definition}

\subsection{Collecting Adversarial Interaction Properties}
\label{subsec:collecting-semantics}

All possible behaviors of a protocol $\pi$ interacting with an adversary $\mathcal{A}$ can be captured as a set of forward-reachable adversarial interaction properties $P$.
This set is defined using a collecting semantics (Figure~\ref{fig:collecting-semantics}), denoted $\collect{\mathcal{A},\pi}$, that is defined with respect to a set of initial system states $P_0$ and set of external channels $\mathcal{C}$.
Note that the system states $\langle \alpha, \kappa, N, \Sigma \rangle$ and adversarial interaction properties $(\alpha, \kappa, N, \Sigma)$ contain the same pieces of information and can be easily converted between.
The collecting semantics is precisely defined as the least fixed point of the set of properties reachable via the system step relation (Section~\ref{subsec:system-semantics}) starting from the given initial system states.
A least fixed point exists by the Knaster-Tarski fixed point theorem since the set of adversarial interaction properties forms a complete lattice under set inclusion, and the $reachNext$ function used below is monotonic.

\begin{figure}[ht]
  \caption{Collecting Semantics for \lang Protocols and Adversaries}
  \label{fig:collecting-semantics}
  \begin{mathpar}
    reachNext ~\mathcal{C} ~\mathcal{A} ~\pi ~P_0 = \lambda~ P.~ P_0
    ~\cup~ P
    ~\cup~
    \left\{
      (\alpha', \kappa', N', \Sigma')
      ~\left|~
      \begin{array}{l}
        \mathcal{C}, \mathcal{A}, \pi \vdash \langle \alpha, \kappa, N, \Sigma \rangle \systemstep \langle \alpha', \kappa', N', \Sigma' \rangle
        \\ \land~
        (\alpha, \kappa, N, \Sigma) \in P
      \end{array}
      \right.
    \right\}

    \\

    \collect{\mathcal{A},\pi}
    ~ \mathcal{C} ~P_0
    =
    \text{lfp} \left( reachNext ~\mathcal{C} ~\mathcal{A} ~\pi ~P_0 \right)
  \end{mathpar}
\end{figure}

In general, the set of reachable adversarial interaction properties is unbounded since given even a single term and a tuple constructor, an adversary can generate an unbounded number of distinct terms by repeatedly applying the constructor.
As a result the set of collected properties is typically unbounded and therefore not computable.
Despite this restriction, the collecting semantics still provide a precise semantics upon which to define adversarial equivalence.
The collecting semantics are further useful as the foundation upon which sound over-approximate abstractions can be defined.

\subsection{Adversarial Equivalence}
\label{subsec:adversarial-equivalence}

% TODO(klinvill): The adversary is defined by the cryptographic model. Ideall $\mathcal{A}$ should also capture the term indistinguishability relation, not just the term derivation functions.
Intuitively, two protocols $\Pi_1$ and $\Pi_2$ are adversarially equivalent for adversary $\mathcal{A}$ and external channels $\mathcal{C}$ when any observable behavior and derivable knowledge from one protocol can be matched for the other protocol, and vice versa.
We use the notation $\Pi$ to refer to the packaged set of initial system states for a protocol, $P_0$, as well as the protocol value that contains the handlers for the protocol, $\pi$.
The intuition for this matching in one direction (from protocol $\Pi_1$ to protocol $\Pi_2$) is formalized in the adversarial preorder relation (Definition~\ref{def:adversarial-preorder}), while adversarial equivalence (Definition~\ref{def:adversarial-equivalence}) simply requires the preorder to hold in both directions.

Adversarial preorder between $\Pi_1$ and $\Pi_2$ holds when the exact same sequence of adversary interactions and observable effects $\alpha$ from $\Pi_1$ can be produced by $\Pi_2$, and the adversary's knowledge after those interactions are statically equivalent.
The possible interactions, effects, and knowledge are captured by the set of forward-reachable adversarial interaction properties produced by the collecting semantics $\collect{\mathcal{A},\pi}$.
The static equivalence condition captures the intuition that while an adversary can unambiguously determine how it interacted with a protocol, it may not be able to distinguish between different pieces of information it has derived from those interactions dependent on the model of cryptographic primitives.
In particular, to model semantically secure encryption schemes, two encrypted messages are considered statically indistinguishable to an adversary, regardless of their contents.

The definition relies on a static equivalence relation $\sim_\kappa$ between adversary knowledges $\kappa_1$ and $\kappa_2$, which is just the pairwise lifting of the indistinguishability relation $\sim_t$ on terms defined by the cryptographic model.
Specifically, we say $\kappa_1 \sim_\kappa \kappa_2$ holds if and only if $\domain{\kappa_1} = \domain{\kappa_2}$ and for every index $i$, if $i \in \domain{\kappa_1}$ then $\kappa_1[i] \sim_t \kappa_2[i]$.
We overload the notation $\sim$ to refer to either $\sim_\kappa$ or $\sim_t$ depending on the context.

\begin{definition}[Adversarial Preorder]
  \label{def:adversarial-preorder}
  A protocol $\Pi_1$ $(\mathcal{A}, \mathcal{C})$-precedes $\Pi_2$, or equivalently $(\pi_1, P_{1,0})$ $(\mathcal{A}, \mathcal{C})$-precedes $(\pi_2, P_{2,0})$, denoted $\Pi_1 \testpreorder[\mathcal{A},\mathcal{C}] \Pi_2$, if for every forward-reachable adversarial interaction property $(\alpha, \kappa_1, N_1, \Sigma_1) \in \left( \collect{\mathcal{A}, \pi_1} ~\mathcal{C} ~P_{1, 0} \right)$ there exists a forward-reachable adversarial interaction property $(\alpha, \kappa_2, N_2, \Sigma_2) \in \left( \collect{\mathcal{A}, \pi_2} ~\mathcal{C} ~P_{2, 0} \right)$ such that $\kappa_1 \sim \kappa_2$.
\end{definition}

Since the adversarial preorder relation is parametric on the static equivalence relation $\sim_\kappa$ for the given cryptographic model, it is only a preorder if $\sim_\kappa$ is also a preorder.
In practice, $\sim_\kappa$ should be an equivalence relation.

\begin{lemma}
  \label{lem:adversarial-preorder}
  $\testpreorder[\mathcal{A}, \mathcal{C}]$ is a preorder (reflexive and transitive), if $\sim_\kappa$ is a preorder.
\end{lemma}
\begin{proof}
  $\testpreorder[\mathcal{A}, \mathcal{C}]$ is trivially reflexive since $\sim_\kappa$ is reflexive (by virtue of being a preorder).
  To show $\testpreorder[\mathcal{A}, \mathcal{C}]$ is also transitive, consider protocols $\Pi$, $\Pi'$, and $\Pi''$ such that $\Pi \testpreorder[\mathcal{A}, \mathcal{C}] \Pi'$ and $\Pi' \testpreorder[\mathcal{A}, \mathcal{C}] \Pi''$.
  For every property $(\alpha, \kappa, N, \Sigma) \in \left( \collect{\mathcal{A}, \pi} ~\mathcal{C} ~P_0 \right)$, there must exist some property $(\alpha, \kappa', N', \Sigma') \in \left( \collect{\mathcal{A}, \pi'} ~\mathcal{C} ~P_0' \right)$ such that $\kappa \sim \kappa'$.
  Similarly, for the property $(\alpha, \kappa', N', \Sigma') \in \left( \collect{\mathcal{A}, \pi'} ~\mathcal{C} ~P_0' \right)$, there must exist some property $(\alpha, \kappa'', N'', \Sigma'') \in \left( \collect{\mathcal{A}, \pi''} ~\mathcal{C} ~P_0'' \right)$ such that $\kappa' \sim \kappa''$.
  Because $\sim_\kappa$ is transitive, $\kappa \sim \kappa'$, and $\kappa' \sim \kappa''$, then $\kappa \sim \kappa''$.
  Since $(\alpha, \kappa'', N'', \Sigma'') \in \left( \collect{\mathcal{A}, \pi''} ~\mathcal{C} ~P_0'' \right)$ and $\kappa \sim \kappa''$, then $\Pi \testpreorder[\mathcal{A}, \mathcal{C}] \Pi''$.
\end{proof}

Given the definition of adversarial preorder, adversarial equivalence is simply defined as adversarial preorder in both directions.

\begin{definition}[Adversarial Equivalence]
  \label{def:adversarial-equivalence}
  Two protocols $\Pi_1$ and $\Pi_2$ are \emph{adversarially equivalent} with respect to an adversary $\mathcal{A}$ for external channels $\mathcal{C}$, denoted $\Pi_1 \testequiv[\mathcal{A}, \mathcal{C}] \Pi_2$, if $\Pi_1 \testpreorder[\mathcal{A}, \mathcal{C}] \Pi_2$ and $\Pi_2 \testpreorder[\mathcal{A}, \mathcal{C}] \Pi_1$.
\end{definition}

\begin{lemma}
  $\testequiv[\mathcal{A}, \mathcal{C}]$ is an equivalence relation, if $\sim_\kappa$ is an equivalence relation.
\end{lemma}
\begin{proof}
  When $\sim_\kappa$ is an equivalence relation, then $\testpreorder[\mathcal{A}, \mathcal{C}]$ is reflexive and transitive by Lemma~\ref{lem:adversarial-preorder}.
  The conjunction of preorders is reflexive and transitive, so $\testequiv[\mathcal{A}, \mathcal{C}]$ is reflexive and transitive.
  $\Pi' \testequiv[\mathcal{A}, \mathcal{C}] \Pi$ is defined as $\Pi' \testpreorder[\mathcal{A}, \mathcal{C}] \Pi$ and $\Pi \testpreorder[\mathcal{A}, \mathcal{C}] \Pi'$.
  Since conjunction is symmetric, this is equivalent to $\Pi \testpreorder[\mathcal{A}, \mathcal{C}] \Pi'$ and $\Pi' \testpreorder[\mathcal{A}, \mathcal{C}] \Pi$.
  Therefore $\testequiv[\mathcal{A}, \mathcal{C}]$ is symmetric.
  Since $\testequiv[\mathcal{A}, \mathcal{C}]$ is reflexive, transitive, and symmetric, it is an equivalence relation.
\end{proof}
