% Comment commands used by Evan. Contain boolean flags to either enable/disable all coments, or enable/disable TODOs and jedi statements.
\newboolean{showcomments}
\setboolean{showcomments}{true}
\newboolean{showjedi}
\setboolean{showjedi}{false}
\newboolean{showtodos}
\setboolean{showtodos}{false}
\newcommand{\editcomment}[2][red]{\ifthenelse{\boolean{showcomments}}{{\color{#1}$\bigl[$\bgroup\em #2\egroup$\bigr]$}}{}}
\newcommand{\note}[1]{\editcomment[blue]{#1}}
\newcommand{\todo}[1]{\ifthenelse{\boolean{showtodos}}{\editcomment{{\bf TODO:} #1}}}
% Evan likes to review "jedi" statements, that should capture the essence of a paragraph (i.e. how would Yoda convey the point of a paragraph?). Should be able to just read the jedi statements to get all the important points of a paper.
\newcommand{\jedi}[1]{\ifthenelse{\boolean{showjedi}}{\editcomment[cyan]{#1}}{}}

% Denotes that the work for this text is still in progress
\newcommand{\wip}[1]{{\color{red} \textbf{#1}}}

% Protocol notation commands
\newcommand{\protocol}{\pi}
\newcommand{\protocolSet}{\Pi}
\newcommand{\trace}{\tau}
\newcommand{\traceSet}{\mathcal{T}}
\newcommand{\attacker}{\mathcal{A}}
\newcommand{\step}[4]{\langle #1 \rangle \xrightarrow[#2]{#3} \langle #4 \rangle}
\newcommand{\stepMulti}[3]{\langle #1 \rangle \xrightarrow[#2]{}^* \langle #3 \rangle}
% Non-deterministic versions of the step relations
\newcommand{\stepND}[4]{#1 \xrightarrow[#2]{#3} #4}
\newcommand{\stepMultiND}[3]{#1 \xrightarrow[#2]{}^* #3}
\newcommand{\tape}{\gamma}
\newcommand{\tapeSet}{\Gamma}

% Notation used in language semantics
\newcommand{\protodefeval}{\Downarrow}
\newcommand{\protoeval}{\Downarrow}
\newcommand{\stmteval}{\Downarrow}
\newcommand{\expreval}{\Downarrow}
\newcommand{\systemstep}{\rightarrow}

% Represents internal transitions or state
\newcommand{\internal}{\tau}
% Represents sub-protocol transitions or state
\newcommand{\subproto}{\mu}

% Common notation
\newcommand{\nat}{\mathbb{N}}
\newcommand{\abs}[1]{\left| #1 \right|}
\newcommand{\quoted}[1]{``#1''}
\newcommand{\defeq}{\stackrel{\text{def}}{=}}
\newcommand{\powerset}{\mathcal{P}}
\newcommand{\compose}[2]{#1 \circ #2}
\newcommand{\domain}[1]{\mathrm{dom}(#1)}
% Equivalent of powerset for bags. In longer form `\Bag(X)` would instead be represented as something like `X \rightarrow \nat`.
\newcommand{\Bag}{\mathcal{B}}
% Record restriction. Keep only the given fields of a record
\newcommand{\restrict}[2]{#1 |_{\{ #2 \}}}
% Bag comprehension, analogous to the notation `{x | p(x)}` for sets
\newcommand{\bagcomp}[2]{ \{\{ #1 ~|~ #2 \}\} }
% Alternative within a non-terminal syntax definition
\newcommand{\ALT}{~\mid~}

% Indistinguishability notation
\newcommand{\ind}{\texttt{ind}}
% "Leak Equivalence", essentially just indistinguishability on traces
\newcommand{\LeEq}{\texttt{LeEq}}
\newcommand{\capabilities}{\mathcal{C}}
% Equivalence class notation
\newcommand{\eqclass}[2]{[#1]_{#2}}

% Denotes the set of lists
\newcommand{\lst}[1]{\overline{#1}}
% Denotes the set of coinductive streams
\newcommand{\stream}[1]{#1^\omega}

% Notation for protocol test relations such test simulation, static equivalence, and test equivalence.
\newcommand{\staticequiv}{\approx}
\newcommand{\testsim}[1][T]{\stackrel{\subset}{\sim}_{#1}}
\newcommand{\testpreorder}[1][T]{\stackrel{<}{\sim}_{#1}}
\newcommand{\testequiv}[1][T]{\approx_{#1}}

% Shortcut to represent a protocol composed with a tester (T by default)
\newcommand{\tested}[2][T]{#1 \parallel #2}

% Name of the framework introduced in this paper
\newcommand{\framework}{\texttt{Placeholder}\xspace}

% Applied pi calculus notation
\newcommand{\apisend}[2]{\overline{#1}\langle #2 \rangle}
\newcommand{\apirecv}[2]{#1 (#2)}

% Inference rule notation
\newcommand{\RULE}[3]{\inferrule*[Right=#1]{#2}{#3}}
\newcommand{\rulelabel}[1]{\textrm{\sc{#1}}}
% Denotation of notation
\newcommand{\denote}[1]{\llbracket #1 \rrbracket}
% Collecting semantics notation
\newcommand{\collect}[1]{\{\!\llbracket #1 \rrbracket\!\}}

% List concatenation operator
\newcommand{\lconcat}{\mathbin{++}}

% Renders the same way as in lstlisting
% \newcommand{\lstkw}[1]{{\lstinline[columns=fixed]{#1}}}
\newcommand{\lstkw}[1]{\texttt{#1}}

% Name of the language introduced in this paper
\newcommand{\lang}{\texttt{DYAct}\xspace}

% Symmetric transitions
\newcommand{\symmetric}[1]{\equiv_{#1}}
